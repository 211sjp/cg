\documentclass[a4paper,UTF8]{article}
\usepackage{ctex}
\usepackage[margin=1.25in]{geometry}
\usepackage{color}
\usepackage{graphicx}
\usepackage{amssymb}
\usepackage{amsmath}
\usepackage{amsthm}
%\usepackage[thmmarks, amsmath, thref]{ntheorem}
\theoremstyle{definition}
\newtheorem*{solution}{Solution}
\newtheorem*{prove}{Proof}
\usepackage{multirow}
\usepackage{url}
\usepackage[colorlinks,urlcolor=blue]{hyperref}
\usepackage{enumerate}
\renewcommand\refname{参考文献}
%--

%--
\begin{document}
\title{\textbf{《计算机图形学》 5月报告 }}
\author{学号,姓名,\href{mailto:xxx@xxx.com}{171240511@smail.nju.edu.cn}}
\maketitle

\section{综述}
在老师所给框架代码上进行修改完善,暂时只完成了画布重置,线段绘制、多边形绘制的算法部分和gui的线段绘制.

\section{算法介绍}
\subsection{绘制线段}
\indent DDA算法:\ 基本思想是记下起点,然后让长的一边变量不断加一,短的一边则不断加斜率乘1后近似为int值.
代码参照\cite{rog_2002}.\\

\indent Bresenham算法:\ 基本思想其实和DDA是一样的.只是Bresenham算法通过变形避免了浮点运算.
这样一来,既提高了运算的效率,又避免了浮点数不断累加造成的长线段误差.
所以本质上,Bresenham是DDA算法的优化形式.代码参照\cite{rog_2002}.\\

\indent 能够处理两端点相等的情况.\\
\indent 因为讲义中说不要求像素级一致,所以和伪代码一样只亮一边端点,即线段点的范围$[P_0,P_1)$.

\subsection{绘制多边形}
默认指令中的点是排好序的, 直接调用线段绘制算法按顺序连点.

\subsection{绘制椭圆}
\indent 使用ppt上的中点椭圆算法, 先画出1/4椭圆, 然后对称. 画1/4椭圆时,根据是否到达切线斜率为-1的位置, 来判断选点的方向是基于x还是基于y, 再根据实际椭圆上的点离哪个点更近取近似.\\
\indent 实际代码要先求出中心和长短轴, 然后以中心为原点求点, 再映射回原来的坐标系.\\
\indent 对称的时候考虑了x轴和y轴上的特例, 不然x=0和y=0的点对称后相当于1个点出现了两次.\\
\indent 代码流程参考书上ppt相关部分.

\subsection{绘制曲线}
\indent Bezier,参考\cite{sun_2006}中6.3.3和6.3.4.\\
\indent B-spline,实现参考了\cite{bspline_web} 没有通过首末控制点.

\subsection{平移}
\indent 平移部分参考ppt, 直接令各图形的控制点偏移[dx,dy]即可.

\subsection{旋转}
\indent 旋转部分参考ppt, 直接套用公式, 各图形的控制点相对于旋转中心[x,y]顺时针旋转r度即可.

\subsection{缩放}
\indent 缩放部分参考ppt, 直接套用公式, 控制点相对于缩放中心[x,y]缩放s倍即可.

\subsection{裁剪}
\indent Cohen-Sutherland算法:\ 这个算法的特别之处在于位运算的思想,用4位二进制数表示左右上下四个位置的是否. 接着再分类讨论不同情况下线段裁剪时, 就可以用位运算来判断怎么裁剪线段. 它会重复裁剪直到线段完全在窗口内或者完全在窗口外.
代码参考\cite{cohen_web}.\\
\indent liang-Barskey算法:\ 这个算法不同于Cohen-Sutherland算法, 它希望一步直接将线段裁剪出来, 而不是重复裁剪.
它用$P=P_1+u*(P2-P1)$来表示直线P, 并分情况判断和裁剪窗口相交点处的$u_1,u_2$值, 根据$u_1,u_2$的情况来判断结果.
代码参考\cite{liang_web}和ppt.

\section{系统介绍}
gui部分中目前仍按原始代码用QListWidget记录图元, 能鼠标控制两种算法绘制线段,
多边形的鼠标绘制尚未完成.\\
画笔颜色选择参考了\cite{pyqt5_ch}, 画笔颜色选择应在绘制图元之前\\
重置画布\\
原来的代码由于点击绘制线段和在画布上press mouse都会使id加1,因此图元编号并不连续,对此进行了修改.\\
保存画布的代码参考了\cite{save_canvas}\\
gui绘制多边形也实现了,最后一条边离起始点5$\times$5范围内闭合.\\
图形没完成的会在点击任意按钮时完成\\
平移时按下的点为[x0,y0], 移动时为[x1,y1], 计算出dx,dy并paint.目前边界得松手才会更新\\
编辑时有些操作参数较多,需要点击两次,比如旋转,这时如果点击一次就切走做别的,第一次点击选择的点仍然会保留在对应item中.但是每次进行编辑操作前都会调用selectedTransClear()函数,确保选择的item的相应参数恢复到初始状态.\\
缩放基本的实现思路和旋转一致.\\
裁剪后,如果线段什么点也没剩下,仍然保留QListWiget中该图元,但是无法对其进行任何操作\\
\section{总结}
暂时防忘:
\indent 1.画布的宽是[0, width-1],高是[0, height-1]\\

\indent 2.gui中sys.exit(app.exec\_())改为app.exec\_(),自己手动退出,不然Spyder运行手动退出时会报错.
此外,还得加上del app,不然重新运行的时候也会有点问题.
参考\cite{exit}\\

\indent 3.框架代码是鼠标一动就将MyItem中的图元绘制,
这导致绘制直线的时候直线随着拖动不断绘制,这包括了最开始端点重合的情况,
对于DDA会出现除0错误,因此要特判.\\

\indent 画笔颜色选择分两步,一个是会用QColorDialog调出调色版,另一个是要将画笔颜色传给MyItem,我选择用全局变量g\_penColor.\\

\bibliographystyle{plain}%
%"xxx" should be your citing file's name.
\bibliography{cgref}
\end{document}